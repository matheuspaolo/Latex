\setcounter{topnumber}{5}
\setcounter{bottomnumber}{5}
\setcounter{totalnumber}{5}

\chapter{Procedimentos e resultados}

\section{Tarefa I}
\subsection{Procedimento}
\begin{description}
  \item[a)]Para iniciar a simula\c{c}\~{a}o de circuitos com o Multisim sugere-se que seja simulado um circuito com tens\~{a}o cont\'{\i}nua e resistores, mostrado na figura 1. Simule o circuito e obtenha os valores solicitados na tabela.\\

      *********figura 1 ************ \\

      Inicialmente simular o circuito da figura 2 e verificar se o diodo est\'{a} em condu\c{c}\~{a}o, al\'{e}m de determinar as grandezas solicitadas na tabela 2. \\

      ******** figura 2 *********** \\


  \item[b)]A seguir, simule o circuito da figura 3, no qual o diodo deve estar bloqueado. Verifique se isto \'{e} verdadeiro e, al\'{e}m disso, anote as grandezas solicitadas na tabela 3. \\

      ******* figura 3 ********* \\
\end{description}

\subsection{Resultado}
\begin{description}
  \item[a)]A simula\c{c}\~{a}o foi realizada usando o software Multisim, como segue na figura.\\
  ************* figura 4 ******** \\
  A tabela foi constru\'{\i}da com base dos valores da simula\c{c}\~{a}o. \\

  *********** tabela 1 ************ \\

  A pr\'{o}xima tabela refere-se aos dados obtidos com a constru\c{c}\~{a}o do circuito 1. Tal constru\c{c}\~{a}o foi realizada com o aux\'{\i}lio de uma protoboard.
Nota-se que os valores obtidos da corrente de todos os elementos foi o mesmo, isso se deu porque tratava-se de um circuito em s\'{e}rie. A pot\^{e}ncia foi obtida (tanto na simula\c{c}\~{a}o, quanto no experimento) por meio da formula abaixo.\\

********** f\'{o}rmulas ************** \\

************ tabela 2 ************ \\

Comparando as Tabelas 1 e 2, constatou-se que os valores obtidos foram muito pr\'{o}ximos e est\~{a}o de acordo com o esperado.



  \item[b)]A simula\c{c}\~{a}o do circuito 2 foi realizada usando o software Multisim, como segue na figura. O diodo em quest\~{a}o est\'{a} polarizado, ou seja, h\'{a} condu\c{c}\~{a}o de corrente. \\

      ********* figura 5 ********** \\
      Os dados contidos na Tabela 3 foram obtidos com a simula\c{c}\~{a}o do circuito.  \\

      ******** tabela 3 ********* \\

      Na Tabela 4 encontram-se os dados obtidos com a constru\c{c}\~{a}o do circuito 2. Tal constru\c{c}\~{a}o foi realizada com o aux\'{\i}lio de uma protoboard. \\

      *********** tabela 4 ********* \\

      \'{E} valido ressaltar neste circuito que os valores da corrente s\~{a}o os mesmo, apesar de o circuito possuir uma resist\^{e}ncia em paralelo, isso se deu porque a corrente sempre “procura” o lugar de menor resist\^{e}ncia para passar, logo, h\'{a} um desvio insignificante de corrente. Ao comparar Tabelas 3 e 4, nota-se que os dados obtidos foram satisfat\'{o}rios ao experimento.

  \item[c)]A simula\c{c}\~{a}o do circuito 3 foi realizada usando o software Multisim, como segue na figura. O diodo encontra-se no estado bloqueado, ou seja, n\~{a}o h\'{a} condu\c{c}\~{a}o de corrente. \\

      ********** figura 6 *********** \\

      Os dados abaixo, da tabela 3, foram obtidos com a simula\c{c}\~{a}o do circuito 3. \\

      ********* tabela 5 ********* \\

      Na tabela abaixo encontram-se os dados obtidos com a constru\c{c}\~{a}o do circuito 3. Tal constru\c{c}\~{a}o foi realizada com o aux\'{\i}lio de uma protoboard. \\

      ********* tabela 6 ********* \\

      Na tabela 5, temos os valores de corrente na ordem de 10-9 ou seja, \'{e} um valor muito pequeno, e experimentalmente na tabela 6, obteve-se tal valor igual a 0, isso se deu devido a precis\~{a}o do Mult\'{\i}metro utilizado. No mais, os valores obtidos foram satisfat\'{o}rios.
\end{description}

\section{Tarefa II}
\subsection{Procedimento}
Realizar todas as atividades simuladas e compar\'{a}-las com os valores experimentais.

\begin{description}
  \item[a)]Simule o circuito retificador de meia onda mostrado na figura 4 e anote os valores solicitados na tabela 4. \\

      ********* figura 7 ********* \\

  \item[b)]Desenhe as formas de onda da tens\~{a}o na entrada do retificador (fonte) e ap\'{o}s o diodo, ou seja, na carga. \\


      ********** falta um gr\'{a}fico aqui ************
\end{description}

\subsection{Resultado}
\begin{description}
  \item[a)]A simula\c{c}\~{a}o foi realizada usando o software Multisim, como segue na figura: \\

  ********** figura 8 ************* \\

  Para a constru\c{c}\~{a}o da Tabela 7, fez-se uso dos dados da Figura 8. \\

  ********** tabela 7 ********** \\

  Ap\'{o}s a constru\c{c}\~{a}o do circuito, com aux\'{\i}lio do oscilosc\'{o}pio fez-se as seguintes medidas (Figura 9) e montou-se a tabela 8. E a medi\c{c}\~{a}o da correte foi feita usando o mult\'{\i}metro. \\

  ********* figura 9 *********** \\

  ******** tabela 8 ************ \\

  Comparando os dados da tabela, observa-se que os resultados foram satisfat\'{o}rios. E as discrep\^{a}ncias existentes s\~{a}o devido \`{a} pouca precis\~{a}o do equipamento ou a complica\c{c}\~{o}es no manuseio.
  \item[b)]As formas de ondas de sa\'{\i}da s\~{a}o similares, como representado nas imagens a seguir:\\
  
  ********* figura 10 *********** \\
  ********* figura 11 *********** \\
  
  Portanto, obteve-se o resultado esperado. 
\end{description}


\section{Tarefa III}
\subsection{Procedimento}
\subsection{Resultado}

\section{Tarefa IV}
\subsection{Procedimento}
\subsection{Resultado}

\section{Tarefa V}
\subsection{Procedimento}
Aplicar na bobina primaria do transformador um sinal com o gerador de fun\c{c}\~{o}es com frequ\^{e}ncia de 60Hz, 1kHz e 10kHz, 100kHz com amplitude de 3 Volts, me\c{c}a os valores de amplitude de tens\~{a}o na bobina secundaria do transformador e preencha a tabela 7 (comente os resultados obtidos).
\subsection{Resultado}
Primeiramente, o gerador de fun\c{c}\~{o}es foi ligado diretamente no oscilosc\'{o}pio, para que a tens\~{a}o e frequ\^{e}ncia pudessem ser ajustadas de forma precisa para os valores desejados. Ap\'{o}s ajustados os valores, o gerador de fun\c{c}\~{o}es foi ligado na entrada da bobina prim\'{a}ria, e a sa\'{\i}da da bobina secund\'{a}ria foi ligada no oscilosc\'{o}pio para a an\'{a}lise de onda. O mesmo procedimento foi realizado para todas as frequ\^{e}ncias desejadas. Com a bobina secund\'{a}ria ligada no oscilosc\'{o}pio, foram medidas as tens\~{o}es para cada uma das frequ\^{e}ncias abaixo e para as formas de ondas senoidal e quadrada: \\

*********** tabela 1 ************** \\

Percebe-se que os valores de tens\~{a}o resultante para todas as frequ\^{e}ncias est\~{a}o bastante pr\'{o}ximos, e que todas as frequ\^{e}ncias resultantes s\~{a}o muito pr\'{o}ximas de 60 Hz. Logo, fica bem demonstrado o funcionamento do transformador, o qual para todas as formas de entrada na bobina prim\'{a}ria, gerou uma mesma sa\'{\i}da na bobina secund\'{a}ria. 