\chapter*[Introdu\c{c}\~{a}o]{Introdu\c{c}\~{a}o}
\addcontentsline{toc}{chapter}{Introdu\c{c}\~{a}o}
Os circuitos retificadores servem para a convers\~{a}o de corrente alternada em corrente cont\'{\i}nua. Utiliza-se para este processo elementos semicondutores, tais como os diodos e transistores, al\'{e}m de um transformador, que pode ser simples ou com deriva\c{c}\~{a}o central. Os experimentos a serem detalhados a seguir foram realizados mediante a utiliza\c{c}\~{a}o de software simulador de circuitos el\'{e}tricos (Multsim), que foram em seguida comparados aos valores experimentais em laborat\'{o}rio. Compreender todos os est\'{a}gios envolvidos em um circuito retificador \'{e} de extrema import\^{a}ncia para o processo de gera\c{c}\~{a}o e aproveitamento da energia el\'{e}trica, posto que grande parte dos componentes eletr\^{o}nicos modernos funcionam apenas alimentados por corrente cont\'{\i}nua. 