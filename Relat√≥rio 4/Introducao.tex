\chapter*[Introdu\c{c}\~{a}o]{Introdu\c{c}\~{a}o}
\addcontentsline{toc}{chapter}{Introdu\c{c}\~{a}o}
Em sua maioria, a energia elétrica residencial disponível no Brasil, apresenta-se sob a forma de corrente alternada senoidal, podendo ser de 220V ou 110V e frequência 60 Hz, logo, como a maioria dos eletrônicos fazem o uso da corrente continua, é necessário realizar uma transformação. E para tal, esses eletrônicos possuem uma fonte que garante a polarização correta para o bom funcionamento do dispositivo, e em se tratando de pequena escala, essa conversão é feita por circuito retificadores.

Esses circuitos retificadores são constituídos basicamente por diodos, que são componentes não lineares que quando ligados, permitem passagem de corrente somente em um sentido, com a exceção do diodo Zener, também relatado nesse experimento.  Num circuito retificador de meia onda há a remoção de metade do sinal de entrada, ou seja, não é tão eficiente quanto o circuito retificador de onda completa, onde a tensão pulsada na saída tem o duas vezes a frequência do sinal de entrada.   