\setcounter{topnumber}{5}
\setcounter{bottomnumber}{5}
\setcounter{totalnumber}{5}

\chapter{Procedimentos e resultados}

\begin{table}[htb]
	\IBGEtab{%
		\caption{Um Exemplo de tabela alinhada que pode ser longa
			ou curta, conforme padr\~{a}o IBGE.}%
		\label{tabela-ibge}
	}{%
		\begin{tabular}{cccc}
			\toprule
			Variável & Valor teórico & Valor prático & Erro (\%) \\
			\midrule \midrule
			$V_{CE}$ & 11/11/1111 & 14,21 V & erer \\
			\midrule
			$V_{BE}$ & 11/11/2111 & 0,629 V & wrwe \\
			\midrule
			$V_{BC}$ & 05/04/1891 & 13,55 V & qeqwe \\
			\midrule
			$I_{C}$ & 05/04/1891 & 10,15 mA & qeqwe \\
			\midrule
			$I_{B}$ & 05/04/1891 & 31,3 $\mu A$ & qeqwe \\
			\midrule
			$I_{E}$ & 05/04/1891 & 10,15 mA & qeqwe \\
			\midrule
			$\beta$ & 05/04/1891 & 10,17 mA & qeqwe \\
			\midrule
			\bottomrule
		\end{tabular}%
	}{%
		\fonte{Produzido pelos autores.}%
		\nota{Esta \'{e} uma nota, que diz que os dados s\~{a}o baseados na
			regress\~{a}o linear.}%
		\nota[Anota\c{c}\~{o}es]{Uma anota\c{c}\~{a}o adicional, que pode ser seguida de v\'{a}rias
			outras.}%
	}
\end{table}