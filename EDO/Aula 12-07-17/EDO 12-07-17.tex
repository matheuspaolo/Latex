%% abtex2-modelo-artigo.tex, v-1.9.6 laurocesar
%% Copyright 2012-2016 by abnTeX2 group at http://www.abntex.net.br/ 
%%
%% This work may be distributed and/or modified under the
%% conditions of the LaTeX Project Public License, either version 1.3
%% of this license or (at your option) any later version.
%% The latest version of this license is in
%%   http://www.latex-project.org/lppl.txt
%% and version 1.3 or later is part of all distributions of LaTeX
%% version 2005/12/01 or later.
%%
%% This work has the LPPL maintenance status `maintained'.
%% 
%% The Current Maintainer of this work is the abnTeX2 team, led
%% by Lauro César Araujo. Further information are available on 
%% http://www.abntex.net.br/
%%
%% This work consists of the files abntex2-modelo-artigo.tex and
%% abntex2-modelo-references.bib
%%

% ------------------------------------------------------------------------
% ------------------------------------------------------------------------
% abnTeX2: Modelo de Artigo Acadêmico em conformidade com
% ABNT NBR 6022:2003: Informação e documentação - Artigo em publicação 
% periódica científica impressa - Apresentação
% ------------------------------------------------------------------------
% ------------------------------------------------------------------------

\documentclass[
	% -- opções da classe memoir --
	article,			% indica que é um artigo acadêmico
	11pt,				% tamanho da fonte
	oneside,			% para impressão apenas no recto. Oposto a twoside
	a4paper,			% tamanho do papel. 
	% -- opções da classe abntex2 --
	%chapter=TITLE,		% títulos de capítulos convertidos em letras maiúsculas
	%section=TITLE,		% títulos de seções convertidos em letras maiúsculas
	%subsection=TITLE,	% títulos de subseções convertidos em letras maiúsculas
	%subsubsection=TITLE % títulos de subsubseções convertidos em letras maiúsculas
	% -- opções do pacote babel --
	english,			% idioma adicional para hifenização
	brazil,				% o último idioma é o principal do documento
	sumario=tradicional
	]{abntex2}


% ---
% PACOTES
% ---

% ---
% Pacotes fundamentais 
% ---
\usepackage{lmodern}			% Usa a fonte Latin Modern
\usepackage[T1]{fontenc}		% Selecao de codigos de fonte.
\usepackage[utf8]{inputenc}		% Codificacao do documento (conversão automática dos acentos)
\usepackage{indentfirst}		% Indenta o primeiro parágrafo de cada seção.
\usepackage{color}				% Controle das cores
\usepackage{graphicx}			% Inclusão de gráficos
\usepackage{microtype} 			% para melhorias de justificação
\usepackage{amsmath}
% ---
		
% ---
% Pacotes adicionais, usados apenas no âmbito do Modelo Canônico do abnteX2
% ---
\usepackage{lipsum}				% para geração de dummy text
% ---
		
% ---
% Pacotes de citações
% ---
\usepackage[brazilian,hyperpageref]{backref}	 % Paginas com as citações na bibl
\usepackage[alf]{abntex2cite}	% Citações padrão ABNT
% ---

% ---
% Configurações do pacote backref
% Usado sem a opção hyperpageref de backref
\renewcommand{\backrefpagesname}{Citado na(s) página(s):~}
% Texto padrão antes do número das páginas
\renewcommand{\backref}{}
% Define os textos da citação
\renewcommand*{\backrefalt}[4]{
	\ifcase #1 %
		Nenhuma citação no texto.%
	\or
		Citado na página #2.%
	\else
		Citado #1 vezes nas páginas #2.%
	\fi}%
% ---

% ---
% Informações de dados para CAPA e FOLHA DE ROSTO
% ---
\titulo{Modelo Canônico de\\ Artigo científico com \abnTeX}
\autor{Equipe \abnTeX\thanks{\url{http://www.abntex.net.br/}} \and Lauro
César
Araujo\thanks{laurocesar@laurocesar.com}}
\local{Brasil}
\data{2015, v-1.9.6}
% ---

% ---
% Configurações de aparência do PDF final

% alterando o aspecto da cor azul
\definecolor{blue}{RGB}{41,5,195}

% informações do PDF
\makeatletter
\hypersetup{
     	%pagebackref=true,
		pdftitle={\@title}, 
		pdfauthor={\@author},
    	pdfsubject={Modelo de artigo científico com abnTeX2},
	    pdfcreator={LaTeX with abnTeX2},
		pdfkeywords={abnt}{latex}{abntex}{abntex2}{atigo científico}, 
		colorlinks=true,       		% false: boxed links; true: colored links
    	linkcolor=blue,          	% color of internal links
    	citecolor=blue,        		% color of links to bibliography
    	filecolor=magenta,      		% color of file links
		urlcolor=blue,
		bookmarksdepth=4
}
\makeatother
% --- 

% ---
% compila o indice
% ---
\makeindex
% ---

% ---
% Altera as margens padrões
% ---
\setlrmarginsandblock{3cm}{3cm}{*}
\setulmarginsandblock{3cm}{3cm}{*}
\checkandfixthelayout
% ---

% --- 
% Espaçamentos entre linhas e parágrafos 
% --- 

% O tamanho do parágrafo é dado por:
\setlength{\parindent}{1.3cm}

% Controle do espaçamento entre um parágrafo e outro:
\setlength{\parskip}{0.2cm}  % tente também \onelineskip

% Espaçamento simples
\SingleSpacing

% ----
% Início do documento
% ----
\begin{document}

% Seleciona o idioma do documento (conforme pacotes do babel)
%\selectlanguage{english}
\selectlanguage{brazil}

% Retira espaço extra obsoleto entre as frases.
\frenchspacing 

% ----------------------------------------------------------
% ELEMENTOS PRÉ-TEXTUAIS
% ----------------------------------------------------------

%---
%
% Se desejar escrever o artigo em duas colunas, descomente a linha abaixo
% e a linha com o texto ``FIM DE ARTIGO EM DUAS COLUNAS''.
% \twocolumn[    		% INICIO DE ARTIGO EM DUAS COLUNAS
%
%---
% página de titulo
\maketitle

% resumo em português
\begin{resumoumacoluna}
Neste trabalho, são apresentados problemas de modelagem matemática sugeridos pelo livro Equações Diferenciais Ordinárias, do autor Dennis G. Zill. O processo de modelagem é algo que pode ser bastante interdisciplinar, visto que, são utilizados conceitos das mais diversas áreas de estudo e conhecimento para estruturação e resolução do problema.
 
 \vspace{\onelineskip}
 
 \noindent
 \textbf{Palavras-chave}: latex. abntex. editoração de texto.
\end{resumoumacoluna}

\textual

\section{Problema 1.3.22}
Para uma ED linear, um PVI de ordem n é:
Resolver:
\begin{equation*}
a_{n}(x) \frac{d^{n}y}{dx^{n}} + a_{n-1} \frac{d^{n-1}y}{dx^{n-1}} + ... + ... a_{1}(x) \frac{dy}{dx} + a_{0}(x)y = g(x)
\end{equation*}

Sujeita a:
\begin{equation}\label{eq1}
y(x_{0}) = y_{0}, y'(x_{0}) = y_{1}, ... , y^{n-1}(x_{0}) = y_{n-1} 
\end{equation}

Obs: Resolver o PVI \eqref{eq1} é procurar uma finção definida em algum intervalo $I$, contendo $x_{0}$ que satisfaça a ED e os $n$ condições iniciais especificadas em $x_{0}$.

O Teorema a seguir dá condições suficientes para a existência de uma única solução para \eqref{eq1}.

\subsection{Teorema 1 (existência e unicidade)}
Seham $a_{n}(x), a_{n-1}(x), ... , a_{0}(x)$ e $g(x)$ contínuas em um intervalo $I$ e seja $a_{n}(x)$ diferente de $0$ para todo $x_{0}$ pertencente a $I$.
Se $x = x_{0}$ for um ponto qualquer nesse intervalo, então existe uma única solução $y(x)$ do PVI \eqref{eq1} nesse intervalo.

\subsubsection{Exemplo 1}
O PVI:
\begin{equation}\label{exemplo1}
3y''' + 5y'' + y' + 7y = 0
\end{equation} 
\begin{equation}\label{exemplo1.2}
y(1) = 0, y'(1) = 0, y''(1) = 0
\end{equation}

possui a solução trivial $y=0$, uma vez que a ED é de 3º ordem, é linear e possui todas os coeficientes constantes, isto é, contínuas, ou seja, o Teorema 1 está satisfeito. Logo, a soluçãp $y = 0 $ do PVI é única em todo intervalo $I$, tal que 0 pertence a $I$.

\subsubsection{Exemplo 2}
A função $y=3e^{2x}+ e^{-2x} -3x$ é uma solução do PVI:
\begin{equation}\label{key}
y'' - 4y = 12x
y(0) = 4 e y'(0) = 1.
\end{equation}

Notemos que a EDO é linear de ordem 2, além disso, os coeficientes são constantes, logo são contínuas. $g(x) - 12x$ é contínua para todo $x$ pertencente aos reais e $a_{2}(x)$ é diferente de 0 sobre todo intervalo contendo $x_{0} = 0$.
Portanto, pelo Teorema 1, temos que $y=3e^{2x} + e^{-2x} -3x$ é a única solução do PVI.
Obs: As hipóteses do Teorema 1, de que $a_{i}(x)..........$ sejam contínuas e $xxxxx$ diferente de 0 para todo $x$ pertencente a I, então a solução do PVI pode não ser única.

\subsubsection{Exemplo 3}
A função $y = cx^{2} + x + 3$ é uma solução do PVI:
\begin{equation*}\label{ex3}
x^{2}y'' - 2xy' +2y = 6
y(0) = 3, y'(0) = 1
\end{equation*}

no intervalo (- infinito, + infinito). Porém, para $x = 0$ $a_{2}(0) = 0^{2} = 0$, isto é, o coeficiente $a_{2}(x)$ não está dentro das hipóteses do Teorema !, isto é, a função $y = cx^2+x+3$ não é a única solução para o PVI dado. Observamos que para qualquer $c pertencence a R$ tomado, iremos obter uma solução diferente para o PVI. Portanto, se as hipóteses do Teorema 1 não são satisfeitas, não teremos a garantia de unicidade da solução.

\section{Equações homogêneas}
Uma EDO LINEAR DE ORDEM $n$ de forma $a_n(x)\dfrac{d^ny}{dx^(n)}$

%\begin{equation}\label{key}
$a_n(x)\dfrac{d^ny}{dx^(n)}=a_{n-1}(x)\dfrac{d^{n-1}y}{dx^{n-1}}+...+a_1(x)\dfrac{dy}{dx}+a_0(x)y=0$ (2)

%\end{equation}
É chamado de equação homogênea, e a EDO
$a_n(x)\dfrac{d^ny}{dx^(n)}=a_{n-1}(x)\dfrac{d^{n-1}y}{dx^{n-1}}+...+a_1(x)\dfrac{dy}{dx}=g(x)$ (3)
chamada de não-homogênea.
Obs.: Deveremos que para resolver uma EDO linear não-homogênea (3), precisamos primeiramente ser capazes de resolver a equação homogênea associada (2).

(2) Daqui para frente vamos sempre considerar que:
\begin{enumerate}
	\item Os coeficientes $a_i(x), i=0,1,2,...,n$ são contínuas.
	\item $g(x)$ é contínua.
	\item $a_n(x) \neq 0$ para todo $x$ no intervalo.
\end{enumerate}

\section{Operadores Lineares}
Frequentemente usa-se o símbolo D quando se faz uma diferenciação por exemplo $$\dfrac{dy}{dx}=D_y$$ O símbolo D é chamado (sub)operador diferencial(sub) uma vez que transforma uma função diferencial em outra função.

Por exemplo: $$D(cos(4x))=-4sen(4x), D(5x^3-6x^2)=15x^2-12x$$

Dessa forma, derivadas de ordem superior o podem ser expressos em formas de D de uma forma natural.

$$\dfrac{d}{dx}(\dfrac{dy}{dx})=\dfrac{d^2y}{dx^2}=D(Dy)=D^2y$$

em geral temos que $\dfrac{d^ny}{dx^n}=D^ny$, onde $y$ é uma função (sub)suficientemente diferenciável.(sub)

Em geral, definimos um operador diferencial de ordem $n$ como $$L = a_n(x)D^n+a_{n-1}(x)D^{n-1}+...+a_1(x)D+a_0(x) (4)$$ Obs.: Devido a linearidade da diferenciação temos que o operador $L$ dado em (4) é linear.

* Temos EDO linear pode ser expressa em termos de $D$, por exemplo:

A equação diferencial linear $y''+5y'+6y=5x-3$ pode ser escrita como: $$D^2y+5Dy+6y=5x-3$$ ou ainda $$(D^2+5D+6)y=5x-3$$

Logo, usando (4) podemos escrever as EDO's lineares de ordem $n$ (2) e (3) como $$L(y)=0 e L(y)=g(x)$$ respectivamente.
$$L(y)=0(seta ida e volta)a_nD^ny+a$$

\end{document}