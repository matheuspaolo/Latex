\setlength{\absparsep}{18pt} % ajusta o espa\c{c}amento dos par\'{a}grafos do resumo
 \begin{resumo}
 Segundo a \citeonline[3.1-3.2]{NBR6028:2003}, o resumo deve ressaltar o
 objetivo, o m\'{e}todo, os resultados e as conclus\~{o}es do documento. A ordem e a extens\~{a}o
 destes itens dependem do tipo de resumo (informativo ou indicativo) e do
 tratamento que cada item recebe no documento original. O resumo deve ser
 precedido da refer\^{e}ncia do documento, com exce\c{c}\~{a}o do resumo inserido no
 pr\'{o}prio documento. (\ldots) As palavras-chave devem figurar logo abaixo do
 resumo, antecedidas da express\~{a}o Palavras-chave:, separadas entre si por
 ponto e finalizadas tamb\'{e}m por ponto.

 \noindent
 \textbf{Palavras-chaves}: latex. abntex. editora\c{c}\~{a}o de texto.
\end{resumo}