\setcounter{topnumber}{5}
\setcounter{bottomnumber}{5}
\setcounter{totalnumber}{5}

\chapter{Procedimentos e resultados}

\section{Tarefa IV}
\subsection{Procedimento}
Explique o funcionamento do diodo Zener?
\subsection{Resultado}
Alguns diodos possuem a característica especial de operação na região de ruptura, onde grandes variações de corrente resultam em pequenas variações de tensão, a esse dispositivo chamamos Diodo Zener. 
É importante ressaltar que quando polarizado diretamente, ele atua como um diodo comum, conduzindo a partir de 0,7 V e ao ser inversamente polarizado, atua como zener.
Suas principais características elétricas são:
\begin{itemize}
	\item A tensão Zener é especificada pelo fabricante e geralmente abreviada à VZ;
	\item A corrente mínima de operação do Zener na região de ruptura é IKZ;
	\item A corrente máxima para o trabalho do Zener é IZM e se ultrapassado tal valor, o diodo será destruído;
	\item A PW é a potência máxima dissipada pelo Zener (No diodo usado no experimento, pode ser dissipado até 1W);
	\item A RZ é a resistência do Zener;
	\item VZ é a tolerância de tensão do diodo Zener.
\end{itemize}

A curva característica do zener é mostrada na figura abaixo:

******* f1 gisele ********* \\

Abaixo a simbologia usual do diodo Zener:

******* f2 gisele **********