%% abtex2-modelo-include-comandos.tex, v-1.9.6 laurocesar
%% Copyright 2012-2016 by abnTeX2 group at http://www.abntex.net.br/
%%
%% This work may be distributed and/or modified under the
%% conditions of the LaTeX Project Public License, either version 1.3
%% of this license or (at your option) any later version.
%% The latest version of this license is in
%%   http://www.latex-project.org/lppl.txt
%% and version 1.3 or later is part of all distributions of LaTeX
%% version 2005/12/01 or later.
%%
%% This work has the LPPL maintenance status `maintained'.
%%
%% The Current Maintainer of this work is the abnTeX2 team, led
%% by Lauro C\'{e}sar Araujo. Further information are available on
%% http://www.abntex.net.br/
%%
%% This work consists of the files abntex2-modelo-include-comandos.tex
%% and abntex2-modelo-img-marca.pdf
%%

% ---
% Este cap\'{\i}tulo, utilizado por diferentes exemplos do abnTeX2, ilustra o uso de
% comandos do abnTeX2 e de LaTeX.
% ---

\chapter{Resultados}\label{cap_exemplos}


% ---
\section{Codifica\c{c}\~{a}o dos arquivos: UTF8}
% ---

A codifica\c{c}\~{a}o de todos os arquivos do \abnTeX\ \'{e} \texttt{UTF8}. \'{E} necess\'{a}rio que
voc\^{e} utilize a mesma codifica\c{c}\~{a}o nos documentos que escrever, inclusive nos
arquivos de base bibliogr\'{a}ficas |.bib|.

% ---
\section{Cita\c{c}\~{o}es diretas}
\label{sec-citacao}
% ---

\index{cita\c{c}\~{o}es!diretas}Utilize o ambiente \texttt{citacao} para incluir
cita\c{c}\~{o}es diretas com mais de tr\^{e}s linhas:

\begin{citacao}
As cita\c{c}\~{o}es diretas, no texto, com mais de tr\^{e}s linhas, devem ser
destacadas com recuo de 4 cm da margem esquerda, com letra menor que a do texto
utilizado e sem as aspas. No caso de documentos datilografados, deve-se
observar apenas o recuo \cite[5.3]{NBR10520:2002}.
\end{citacao}

Use o ambiente assim:

\begin{verbatim}
\begin{citacao}
As cita\c{c}\~{o}es diretas, no texto, com mais de tr\^{e}s linhas [...] deve-se observar
apenas o recuo \cite[5.3]{NBR10520:2002}.
\end{citacao}
\end{verbatim}

O ambiente \texttt{citacao} pode receber como par\^{a}metro opcional um nome de
idioma previamente carregado nas op\c{c}\~{o}es da classe (\autoref{sec-hifenizacao}). Nesse
caso, o texto da cita\c{c}\~{a}o \'{e} automaticamente escrito em it\'{a}lico e a hifeniza\c{c}\~{a}o \'{e}
ajustada para o idioma selecionado na op\c{c}\~{a}o do ambiente. Por exemplo:

\begin{verbatim}
\begin{citacao}[english]
Text in English language in italic with correct hyphenation.
\end{citacao}
\end{verbatim}

Tem como resultado:

\begin{citacao}[english]
Text in English language in italic with correct hyphenation.
\end{citacao}

\index{cita\c{c}\~{o}es!simples}Cita\c{c}\~{o}es simples, com at\'{e} tr\^{e}s linhas, devem ser
inclu\'{\i}das com aspas. Observe que em \LaTeX as aspas iniciais s\~{a}o diferentes das
finais: ``Amor \'{e} fogo que arde sem se ver''.

% ---
\section{Notas de rodap\'{e}}
% ---

As notas de rodap\'{e} s\~{a}o detalhadas pela NBR 14724:2011 na se\c{c}\~{a}o 5.2.1\footnote{As
notas devem ser digitadas ou datilografadas dentro das margens, ficando
separadas do texto por um espa\c{c}o simples de entre as linhas e por filete de 5
cm, a partir da margem esquerda. Devem ser alinhadas, a partir da segunda linha
da mesma nota, abaixo da primeira letra da primeira palavra, de forma a destacar
o expoente, sem espa\c{c}o entre elas e com fonte menor
\citeonline[5.2.1]{NBR14724:2011}.}\footnote{Caso uma s\'{e}rie de notas sejam
criadas sequencialmente, o \abnTeX\ instrui o \LaTeX\ para que uma v\'{\i}rgula seja
colocada ap\'{o}s cada n\'{u}mero do expoente que indica a nota de rodap\'{e} no corpo do
texto.}\footnote{Verifique se os n\'{u}meros do expoente possuem uma v\'{\i}rgula para
dividi-los no corpo do texto.}.


% ---
\section{Tabelas}
% ---

\index{tabelas}A \autoref{tab-nivinv} \'{e} um exemplo de tabela constru\'{\i}da em
\LaTeX.

\begin{table}[htb]
\ABNTEXfontereduzida
\caption[N\'{\i}veis de investiga\c{c}\~{a}o]{N\'{\i}veis de investiga\c{c}\~{a}o.}
\label{tab-nivinv}
\begin{tabular}{p{2.6cm}|p{6.0cm}|p{2.25cm}|p{3.40cm}}
  %\hline
   \textbf{N\'{\i}vel de Investiga\c{c}\~{a}o} & \textbf{Insumos}  & \textbf{Sistemas de Investiga\c{c}\~{a}o}  & \textbf{Produtos}  \\
    \hline
    Meta-n\'{\i}vel & Filosofia\index{filosofia} da Ci\^{e}ncia  & Epistemologia &
    Paradigma  \\
    \hline
    N\'{\i}vel do objeto & Paradigmas do metan\'{\i}vel e evid\^{e}ncias do n\'{\i}vel inferior &
    Ci\^{e}ncia  & Teorias e modelos \\
    \hline
    N\'{\i}vel inferior & Modelos e m\'{e}todos do n\'{\i}vel do objeto e problemas do n\'{\i}vel inferior & Pr\'{a}tica & Solu\c{c}\~{a}o de problemas  \\
   % \hline
\end{tabular}
\legend{Fonte: \citeonline{van86}}
\end{table}

J\'{a} a \autoref{tabela-ibge} apresenta uma tabela criada conforme o padr\~{a}o do
\citeonline{ibge1993} requerido pelas normas da ABNT para documentos t\'{e}cnicos e
acad\^{e}micos.

\begin{table}[htb]
\IBGEtab{%
  \caption{Um Exemplo de tabela alinhada que pode ser longa
  ou curta, conforme padr\~{a}o IBGE.}%
  \label{tabela-ibge}
}{%
  \begin{tabular}{ccc}
  \toprule
   Nome & Nascimento & Documento \\
  \midrule \midrule
   Maria da Silva & 11/11/1111 & 111.111.111-11 \\
  \midrule
   Jo\~{a}o Souza & 11/11/2111 & 211.111.111-11 \\
  \midrule
   Laura Vicu\~{n}a & 05/04/1891 & 3111.111.111-11 \\
  \bottomrule
\end{tabular}%
}{%
  \fonte{Produzido pelos autores.}%
  \nota{Esta \'{e} uma nota, que diz que os dados s\~{a}o baseados na
  regress\~{a}o linear.}%
  \nota[Anota\c{c}\~{o}es]{Uma anota\c{c}\~{a}o adicional, que pode ser seguida de v\'{a}rias
  outras.}%
  }
\end{table}


% ---
\section{Figuras}
% ---

\index{figuras}Figuras podem ser criadas diretamente em \LaTeX,
como o exemplo da \autoref{fig_circulo}.

\begin{figure}[htb]
	\caption{\label{fig_circulo}A delimita\c{c}\~{a}o do espa\c{c}o}
	\begin{center}
	    \setlength{\unitlength}{5cm}
		\begin{picture}(1,1)
		\put(0,0){\line(0,1){1}}
		\put(0,0){\line(1,0){1}}
		\put(0,0){\line(1,1){1}}
		\put(0,0){\line(1,2){.5}}
		\put(0,0){\line(1,3){.3333}}
		\put(0,0){\line(1,4){.25}}
		\put(0,0){\line(1,5){.2}}
		\put(0,0){\line(1,6){.1667}}
		\put(0,0){\line(2,1){1}}
		\put(0,0){\line(2,3){.6667}}
		\put(0,0){\line(2,5){.4}}
		\put(0,0){\line(3,1){1}}
		\put(0,0){\line(3,2){1}}
		\put(0,0){\line(3,4){.75}}
		\put(0,0){\line(3,5){.6}}
		\put(0,0){\line(4,1){1}}
		\put(0,0){\line(4,3){1}}
		\put(0,0){\line(4,5){.8}}
		\put(0,0){\line(5,1){1}}
		\put(0,0){\line(5,2){1}}
		\put(0,0){\line(5,3){1}}
		\put(0,0){\line(5,4){1}}
		\put(0,0){\line(5,6){.8333}}
		\put(0,0){\line(6,1){1}}
		\put(0,0){\line(6,5){1}}
		\end{picture}
	\end{center}
	\legend{Fonte: os autores}
\end{figure}

Ou ent\~{a}o figuras podem ser incorporadas de arquivos externos, como \'{e} o caso da
\autoref{fig_grafico}. Se a figura que ser inclu\'{\i}da se tratar de um diagrama, um
gr\'{a}fico ou uma ilustra\c{c}\~{a}o que voc\^{e} mesmo produza, priorize o uso de imagens
vetoriais no formato PDF. Com isso, o tamanho do arquivo final do trabalho ser\'{a}
menor, e as imagens ter\~{a}o uma apresenta\c{c}\~{a}o melhor, principalmente quando
impressas, uma vez que imagens vetorias s\~{a}o perfeitamente escal\'{a}veis para
qualquer dimens\~{a}o. Nesse caso, se for utilizar o Microsoft Excel para produzir
gr\'{a}ficos, ou o Microsoft Word para produzir ilustra\c{c}\~{o}es, exporte-os como PDF e
os incorpore ao documento conforme o exemplo abaixo. No entanto, para manter a
coer\^{e}ncia no uso de software livre (j\'{a} que voc\^{e} est\'{a} usando \LaTeX e \abnTeX),
teste a ferramenta \textsf{InkScape}\index{InkScape}
(\url{http://inkscape.org/}). Ela \'{e} uma excelente op\c{c}\~{a}o de c\'{o}digo-livre para
produzir ilustra\c{c}\~{o}es vetoriais, similar ao CorelDraw\index{CorelDraw} ou ao Adobe
Illustrator\index{Adobe Illustrator}. De todo modo, caso n\~{a}o seja poss\'{\i}vel
utilizar arquivos de imagens como PDF, utilize qualquer outro formato, como
JPEG, GIF, BMP, etc. Nesse caso, voc\^{e} pode tentar aprimorar as imagens
incorporadas com o software livre \textsf{Gimp}\index{Gimp}
(\url{http://www.gimp.org/}). Ele \'{e} uma alternativa livre ao Adobe
Photoshop\index{Adobe Photoshop}.

\begin{figure}[htb]
	\caption{\label{fig_grafico}Gr\'{a}fico produzido em Excel e salvo como PDF}
	\begin{center}
	    \includegraphics[scale=0.5]{abntex2-modelo-img-grafico.pdf}
	\end{center}
	\legend{Fonte: \citeonline[p. 24]{araujo2012}}
\end{figure}

% ---
\subsection{Figuras em \emph{minipages}}
% ---

\emph{Minipages} s\~{a}o usadas para inserir textos ou outros elementos em quadros
com tamanhos e posi\c{c}\~{o}es controladas. Veja o exemplo da
\autoref{fig_minipage_imagem1} e da \autoref{fig_minipage_grafico2}.

\begin{figure}[htb]
 \label{teste}
 \centering
  \begin{minipage}{0.4\textwidth}
    \centering
    \caption{Imagem 1 da minipage} \label{fig_minipage_imagem1}
    \includegraphics[scale=0.9]{abntex2-modelo-img-marca.pdf}
    \legend{Fonte: Produzido pelos autores}
  \end{minipage}
  \hfill
  \begin{minipage}{0.4\textwidth}
    \centering
    \caption{Grafico 2 da minipage} \label{fig_minipage_grafico2}
    \includegraphics[scale=0.2]{abntex2-modelo-img-grafico.pdf}
    \legend{Fonte: \citeonline[p. 24]{araujo2012}}
  \end{minipage}
\end{figure}

Observe que, segundo a \citeonline[se\c{c}\~{o}es 4.2.1.10 e 5.8]{NBR14724:2011}, as
ilustra\c{c}\~{o}es devem sempre ter numera\c{c}\~{a}o cont\'{\i}nua e \'{u}nica em todo o documento:

\begin{citacao}
Qualquer que seja o tipo de ilustra\c{c}\~{a}o, sua identifica\c{c}\~{a}o aparece na parte
superior, precedida da palavra designativa (desenho, esquema, fluxograma,
fotografia, gr\'{a}fico, mapa, organograma, planta, quadro, retrato, figura,
imagem, entre outros), seguida de seu n\'{u}mero de ordem de ocorr\^{e}ncia no texto,
em algarismos ar\'{a}bicos, travess\~{a}o e do respectivo t\'{\i}tulo. Ap\'{o}s a ilustra\c{c}\~{a}o, na
parte inferior, indicar a fonte consultada (elemento obrigat\'{o}rio, mesmo que
seja produ\c{c}\~{a}o do pr\'{o}prio autor), legenda, notas e outras informa\c{c}\~{o}es
necess\'{a}rias \`{a} sua compreens\~{a}o (se houver). A ilustra\c{c}\~{a}o deve ser citada no
texto e inserida o mais pr\'{o}ximo poss\'{\i}vel do trecho a que se
refere. \cite[se\c{c}\~{o}es 5.8]{NBR14724:2011}
\end{citacao}

% ---
\section{Express\~{o}es matem\'{a}ticas}
% ---

\index{express\~{o}es matem\'{a}ticas}Use o ambiente \texttt{equation} para escrever
express\~{o}es matem\'{a}ticas numeradas:

\begin{equation}
  \forall x \in X, \quad \exists \: y \leq \epsilon
\end{equation}

Escreva express\~{o}es matem\'{a}ticas entre \$ e \$, como em $ \lim_{x \to \infty}
\exp(-x) = 0 $, para que fiquem na mesma linha.

Tamb\'{e}m \'{e} poss\'{\i}vel usar colchetes para indicar o in\'{\i}cio de uma express\~{a}o
matem\'{a}tica que n\~{a}o \'{e} numerada.

\[
\left|\sum_{i=1}^n a_ib_i\right|
\le
\left(\sum_{i=1}^n a_i^2\right)^{1/2}
\left(\sum_{i=1}^n b_i^2\right)^{1/2}
\]

Consulte mais informa\c{c}\~{o}es sobre express\~{o}es matem\'{a}ticas em
\url{https://github.com/abntex/abntex2/wiki/Referencias}.

% ---
\section{Enumera\c{c}\~{o}es: al\'{\i}neas e subal\'{\i}neas}
% ---

\index{al\'{\i}neas}\index{subal\'{\i}neas}\index{incisos}Quando for necess\'{a}rio enumerar
os diversos assuntos de uma se\c{c}\~{a}o que n\~{a}o possua t\'{\i}tulo, esta deve ser
subdividida em al\'{\i}neas \cite[4.2]{NBR6024:2012}:

\begin{alineas}

  \item os diversos assuntos que n\~{a}o possuam t\'{\i}tulo pr\'{o}prio, dentro de uma mesma
  se\c{c}\~{a}o, devem ser subdivididos em al\'{\i}neas;

  \item o texto que antecede as al\'{\i}neas termina em dois pontos;
  \item as al\'{\i}neas devem ser indicadas alfabeticamente, em letra min\'{u}scula,
  seguida de par\^{e}ntese. Utilizam-se letras dobradas, quando esgotadas as
  letras do alfabeto;

  \item as letras indicativas das al\'{\i}neas devem apresentar recuo em rela\c{c}\~{a}o \`{a}
  margem esquerda;

  \item o texto da al\'{\i}nea deve come\c{c}ar por letra min\'{u}scula e terminar em
  ponto-e-v\'{\i}rgula, exceto a \'{u}ltima al\'{\i}nea que termina em ponto final;

  \item o texto da al\'{\i}nea deve terminar em dois pontos, se houver subal\'{\i}nea;

  \item a segunda e as seguintes linhas do texto da al\'{\i}nea come\c{c}a sob a
  primeira letra do texto da pr\'{o}pria al\'{\i}nea;

  \item subal\'{\i}neas \cite[4.3]{NBR6024:2012} devem ser conforme as al\'{\i}neas a
  seguir:

  \begin{alineas}
     \item as subal\'{\i}neas devem come\c{c}ar por travess\~{a}o seguido de espa\c{c}o;

     \item as subal\'{\i}neas devem apresentar recuo em rela\c{c}\~{a}o \`{a} al\'{\i}nea;

     \item o texto da subal\'{\i}nea deve come\c{c}ar por letra min\'{u}scula e terminar em
     ponto-e-v\'{\i}rgula. A \'{u}ltima subal\'{\i}nea deve terminar em ponto final, se n\~{a}o
     houver al\'{\i}nea subsequente;

     \item a segunda e as seguintes linhas do texto da subal\'{\i}nea come\c{c}am sob a
     primeira letra do texto da pr\'{o}pria subal\'{\i}nea.
  \end{alineas}

  \item no \abnTeX\ est\~{a}o dispon\'{\i}veis os ambientes \texttt{incisos} e
  \texttt{subalineas}, que em suma s\~{a}o o mesmo que se criar outro n\'{\i}vel de
  \texttt{alineas}, como nos exemplos \`{a} seguir:

  \begin{incisos}
    \item \textit{Um novo inciso em it\'{a}lico};
  \end{incisos}

  \item Al\'{\i}nea em \textbf{negrito}:

  \begin{subalineas}
    \item \textit{Uma subal\'{\i}nea em it\'{a}lico};
    \item \underline{\textit{Uma subal\'{\i}nea em it\'{a}lico e sublinhado}};
  \end{subalineas}

  \item \'{U}ltima al\'{\i}nea com \emph{\^{e}nfase}.

\end{alineas}

% ---
\section{Espa\c{c}amento entre par\'{a}grafos e linhas}
% ---

\index{espa\c{c}amento!dos par\'{a}grafos}O tamanho do par\'{a}grafo, espa\c{c}o entre a margem
e o in\'{\i}cio da frase do par\'{a}grafo, \'{e} definido por:

\begin{verbatim}
   \setlength{\parindent}{1.3cm}
\end{verbatim}

\index{espa\c{c}amento!do primeiro par\'{a}grafo}Por padr\~{a}o, n\~{a}o h\'{a} espa\c{c}amento no
primeiro par\'{a}grafo de cada in\'{\i}cio de divis\~{a}o do documento
(\autoref{sec-divisoes}). Por\'{e}m, voc\^{e} pode definir que o primeiro par\'{a}grafo
tamb\'{e}m seja indentado, como \'{e} o caso deste documento. Para isso, apenas inclua o
pacote \textsf{indentfirst} no pre\^{a}mbulo do documento:

\begin{verbatim}
   \usepackage{indentfirst}      % Indenta o primeiro par\'{a}grafo de cada se\c{c}\~{a}o.
\end{verbatim}

\index{espa\c{c}amento!entre os par\'{a}grafos}O espa\c{c}amento entre um par\'{a}grafo e outro
pode ser controlado por meio do comando:

\begin{verbatim}
  \setlength{\parskip}{0.2cm}  % tente tamb\'{e}m \onelineskip
\end{verbatim}

\index{espa\c{c}amento!entre as linhas}O controle do espa\c{c}amento entre linhas \'{e}
definido por:

\begin{verbatim}
  \OnehalfSpacing       % espa\c{c}amento um e meio (padr\~{a}o);
  \DoubleSpacing        % espa\c{c}amento duplo
  \SingleSpacing        % espa\c{c}amento simples	
\end{verbatim}

Para isso, tamb\'{e}m est\~{a}o dispon\'{\i}veis os ambientes:

\begin{verbatim}
  \begin{SingleSpace} ...\end{SingleSpace}
  \begin{Spacing}{hfactori} ... \end{Spacing}
  \begin{OnehalfSpace} ... \end{OnehalfSpace}
  \begin{OnehalfSpace*} ... \end{OnehalfSpace*}
  \begin{DoubleSpace} ... \end{DoubleSpace}
  \begin{DoubleSpace*} ... \end{DoubleSpace*}
\end{verbatim}

Para mais informa\c{c}\~{o}es, consulte \citeonline[p. 47-52 e 135]{memoir}.

% ---
\section{Inclus\~{a}o de outros arquivos}\label{sec-include}
% ---

\'{E} uma boa pr\'{a}tica dividir o seu documento em diversos arquivos, e n\~{a}o
apenas escrever tudo em um \'{u}nico. Esse recurso foi utilizado neste
documento. Para incluir diferentes arquivos em um arquivo principal,
de modo que cada arquivo inclu\'{\i}do fique em uma p\'{a}gina diferente, utilize o
comando:

\begin{verbatim}
   \include{documento-a-ser-incluido}      % sem a extens\~{a}o .tex
\end{verbatim}

Para incluir documentos sem quebra de p\'{a}ginas, utilize:

\begin{verbatim}
   \input{documento-a-ser-incluido}      % sem a extens\~{a}o .tex
\end{verbatim}

% ---
\section{Compilar o documento \LaTeX}
% ---

Geralmente os editores \LaTeX, como o
TeXlipse\footnote{\url{http://texlipse.sourceforge.net/}}, o
Texmaker\footnote{\url{http://www.xm1math.net/texmaker/}}, entre outros,
compilam os documentos automaticamente, de modo que voc\^{e} n\~{a}o precisa se
preocupar com isso.

No entanto, voc\^{e} pode compilar os documentos \LaTeX usando os seguintes
comandos, que devem ser digitados no \emph{Prompt de Comandos} do Windows ou no
\emph{Terminal} do Mac ou do Linux:

\begin{verbatim}
   pdflatex ARQUIVO_PRINCIPAL.tex
   bibtex ARQUIVO_PRINCIPAL.aux
   makeindex ARQUIVO_PRINCIPAL.idx
   makeindex ARQUIVO_PRINCIPAL.nlo -s nomencl.ist -o ARQUIVO_PRINCIPAL.nls
   pdflatex ARQUIVO_PRINCIPAL.tex
   pdflatex ARQUIVO_PRINCIPAL.tex
\end{verbatim}

% ---
\section{Remiss\~{o}es internas}
% ---

Ao nomear a \autoref{tab-nivinv} e a \autoref{fig_circulo}, apresentamos um
exemplo de remiss\~{a}o interna, que tamb\'{e}m pode ser feita quando indicamos o
\autoref{cap_exemplos}, que tem o nome \emph{\nameref{cap_exemplos}}. O n\'{u}mero
do cap\'{\i}tulo indicado \'{e} \ref{cap_exemplos}, que se inicia \`{a}
\autopageref{cap_exemplos}\footnote{O n\'{u}mero da p\'{a}gina de uma remiss\~{a}o pode ser
obtida tamb\'{e}m assim:
\pageref{cap_exemplos}.}.
Veja a \autoref{sec-divisoes} para outros exemplos de remiss\~{o}es internas entre
se\c{c}\~{o}es, subse\c{c}\~{o}es e subsubse\c{c}\~{o}es.

O c\'{o}digo usado para produzir o texto desta se\c{c}\~{a}o \'{e}:

\begin{verbatim}
Ao nomear a \autoref{tab-nivinv} e a \autoref{fig_circulo}, apresentamos um
exemplo de remiss\~{a}o interna, que tamb\'{e}m pode ser feita quando indicamos o
\autoref{cap_exemplos}, que tem o nome \emph{\nameref{cap_exemplos}}. O n\'{u}mero
do cap\'{\i}tulo indicado \'{e} \ref{cap_exemplos}, que se inicia \`{a}
\autopageref{cap_exemplos}\footnote{O n\'{u}mero da p\'{a}gina de uma remiss\~{a}o pode ser
obtida tamb\'{e}m assim:
\pageref{cap_exemplos}.}.
Veja a \autoref{sec-divisoes} para outros exemplos de remiss\~{o}es internas entre
se\c{c}\~{o}es, subse\c{c}\~{o}es e subsubse\c{c}\~{o}es.
\end{verbatim}

% ---
\section{Divis\~{o}es do documento: se\c{c}\~{a}o}\label{sec-divisoes}
% ---

Esta se\c{c}\~{a}o testa o uso de divis\~{o}es de documentos. Esta \'{e} a
\autoref{sec-divisoes}. Veja a \autoref{sec-divisoes-subsection}.

\subsection{Divis\~{o}es do documento: subse\c{c}\~{a}o}\label{sec-divisoes-subsection}

Isto \'{e} uma subse\c{c}\~{a}o. Veja a \autoref{sec-divisoes-subsubsection}, que \'{e} uma
\texttt{subsubsection} do \LaTeX, mas \'{e} impressa chamada de ``subse\c{c}\~{a}o'' porque
no Portugu\^{e}s n\~{a}o temos a palavra ``subsubse\c{c}\~{a}o''.

\subsubsection{Divis\~{o}es do documento: subsubse\c{c}\~{a}o}
\label{sec-divisoes-subsubsection}

Isto \'{e} uma subsubse\c{c}\~{a}o.

\subsubsection{Divis\~{o}es do documento: subsubse\c{c}\~{a}o}

Isto \'{e} outra subsubse\c{c}\~{a}o.

\subsection{Divis\~{o}es do documento: subse\c{c}\~{a}o}\label{sec-exemplo-subsec}

Isto \'{e} uma subse\c{c}\~{a}o.

\subsubsection{Divis\~{o}es do documento: subsubse\c{c}\~{a}o}

Isto \'{e} mais uma subsubse\c{c}\~{a}o da \autoref{sec-exemplo-subsec}.


\subsubsubsection{Esta \'{e} uma subse\c{c}\~{a}o de quinto
n\'{\i}vel}\label{sec-exemplo-subsubsubsection}

Esta \'{e} uma se\c{c}\~{a}o de quinto n\'{\i}vel. Ela \'{e} produzida com o seguinte comando:

\begin{verbatim}
\subsubsubsection{Esta \'{e} uma subse\c{c}\~{a}o de quinto
n\'{\i}vel}\label{sec-exemplo-subsubsubsection}
\end{verbatim}

\subsubsubsection{Esta \'{e} outra subse\c{c}\~{a}o de quinto n\'{\i}vel}\label{sec-exemplo-subsubsubsection-outro}

Esta \'{e} outra se\c{c}\~{a}o de quinto n\'{\i}vel.


\paragraph{Este \'{e} um par\'{a}grafo numerado}\label{sec-exemplo-paragrafo}

Este \'{e} um exemplo de par\'{a}grafo nomeado. Ele \'{e} produzida com o comando de
par\'{a}grafo:

\begin{verbatim}
\paragraph{Este \'{e} um par\'{a}grafo nomeado}\label{sec-exemplo-paragrafo}
\end{verbatim}

A numera\c{c}\~{a}o entre par\'{a}grafos numeradaos e subsubsubse\c{c}\~{o}es s\~{a}o cont\'{\i}nuas.

\paragraph{Esta \'{e} outro par\'{a}grafo numerado}\label{sec-exemplo-paragrafo-outro}

Esta \'{e} outro par\'{a}grafo nomeado.

% ---
\section{Este \'{e} um exemplo de nome de se\c{c}\~{a}o longo. Ele deve estar
alinhado \`{a} esquerda e a segunda e demais linhas devem iniciar logo abaixo da
primeira palavra da primeira linha}
% ---

Isso atende \`{a} norma \citeonline[se\c{c}\~{o}es de 5.2.2 a 5.2.4]{NBR14724:2011}
 e \citeonline[se\c{c}\~{o}es de 3.1 a 3.8]{NBR6024:2012}.

% ---
\section{Diferentes idiomas e hifeniza\c{c}\~{o}es}
\label{sec-hifenizacao}
% ---

Para usar hifeniza\c{c}\~{o}es de diferentes idiomas, inclua nas op\c{c}\~{o}es do documento o
nome dos idiomas que o seu texto cont\'{e}m. Por exemplo (para melhor
visualiza\c{c}\~{a}o, as op\c{c}\~{o}es foram quebras em diferentes linhas):

\begin{verbatim}
\documentclass[
	12pt,
	openright,
	twoside,
	a4paper,
	english,
	french,
	spanish,
	brazil
	]{abntex2}
\end{verbatim}

O idioma portugu\^{e}s-brasileiro (\texttt{brazil}) \'{e} inclu\'{\i}do automaticamente pela
classe \textsf{abntex2}. Por\'{e}m, mesmo assim a op\c{c}\~{a}o \texttt{brazil} deve ser
informada como a \'{u}ltima op\c{c}\~{a}o da classe para que todos os pacotes reconhe\c{c}am o
idioma. Vale ressaltar que a \'{u}ltima op\c{c}\~{a}o de idioma \'{e} a utilizada por padr\~{a}o no
documento. Desse modo, caso deseje escrever um texto em ingl\^{e}s que tenha
cita\c{c}\~{o}es em portugu\^{e}s e em franc\^{e}s, voc\^{e} deveria usar o pre\^{a}mbulo como abaixo:

\begin{verbatim}
\documentclass[
	12pt,
	openright,
	twoside,
	a4paper,
	french,
	brazil,
	english
	]{abntex2}
\end{verbatim}

A lista completa de idiomas suportados, bem como outras op\c{c}\~{o}es de hifeniza\c{c}\~{a}o,
est\~{a}o dispon\'{\i}veis em \citeonline[p.~5-6]{babel}.

Exemplo de hifeniza\c{c}\~{a}o em ingl\^{e}s\footnote{Extra\'{\i}do de:
\url{http://en.wikibooks.org/wiki/LaTeX/Internationalization}}:

\begin{otherlanguage*}{english}
\textit{Text in English language. This environment switches all language-related
definitions, like the language specific names for figures, tables etc. to the other
language. The starred version of this environment typesets the main text
according to the rules of the other language, but keeps the language specific
string for ancillary things like figures, in the main language of the document.
The environment hyphenrules switches only the hyphenation patterns used; it can
also be used to disallow hyphenation by using the language name
`nohyphenation'.}
\end{otherlanguage*}

Exemplo de hifeniza\c{c}\~{a}o em franc\^{e}s\footnote{Extra\'{\i}do de:
\url{http://bigbrowser.blog.lemonde.fr/2013/02/17/tu-ne-tweeteras-point-le-vatican-interdit-aux-cardinaux-de-tweeter-pendant-le-conclave/}}:

\begin{otherlanguage*}{french}
\textit{Texte en fran\c{c}ais. Pas question que Twitter ne vienne faire une
concurrence d\'{e}loyale \`{a} la traditionnelle fum\'{e}e blanche qui marque l'\'{e}lection
d'un nouveau pape. Pour \'{e}viter toute fuite pr\'{e}coce, le Vatican a donc pris un
peu d'avance, et a d\'{e}j\`{a} interdit aux cardinaux qui prendront part au vote
d'utiliser le r\'{e}seau social, selon Catholic News Service. Une mesure valable
surtout pour les neuf cardinaux – sur les 117 du conclave – pratiquants tr\`{e}s
actifs de Twitter, qui auront interdiction pendant toute la p\'{e}riode de se
connecter \`{a} leur compte.}
\end{otherlanguage*}

Pequeno texto em espanhol\footnote{Extra\'{\i}do de:
\url{http://internacional.elpais.com/internacional/2013/02/17/actualidad/1361102009_913423.html}}:

\foreignlanguage{spanish}{\textit{Decenas de miles de personas ovacionan al pont\'{\i}fice en su
pen\'{u}ltimo \'{a}ngelus dominical, el primero desde que anunciase su renuncia. El Papa se
centra en la cr\'{\i}tica al materialismo}}.

O idioma geral do texto por ser alterado como no exemplo seguinte:

\begin{verbatim}
  \selectlanguage{english}
\end{verbatim}

Isso altera automaticamente a hifeniza\c{c}\~{a}o e todos os nomes constantes de
refer\^{e}ncias do documento para o idioma ingl\^{e}s. Consulte o manual da classe
\cite{abntex2classe} para obter orienta\c{c}\~{o}es adicionais sobre internacionaliza\c{c}\~{a}o de
documentos produzidos com \abnTeX.

A \autoref{sec-citacao} descreve o ambiente \texttt{citacao} que pode receber
como par\^{a}metro um idioma a ser usado na cita\c{c}\~{a}o.

% ---
\section{Consulte o manual da classe \textsf{abntex2}}
% ---

Consulte o manual da classe \textsf{abntex2} \cite{abntex2classe} para uma
refer\^{e}ncia completa das macros e ambientes dispon\'{\i}veis.

Al\'{e}m disso, o manual possui informa\c{c}\~{o}es adicionais sobre as normas ABNT
observadas pelo \abnTeX\ e considera\c{c}\~{o}es sobre eventuais requisitos espec\'{\i}ficos
n\~{a}o atendidos, como o caso da \citeonline[se\c{c}\~{a}o 5.2.2]{NBR14724:2011}, que
especifica o espa\c{c}amento entre os cap\'{\i}tulos e o in\'{\i}cio do texto, regra
propositalmente n\~{a}o atendida pelo presente modelo.

% ---
\section{Refer\^{e}ncias bibliogr\'{a}ficas}
% ---

A formata\c{c}\~{a}o das refer\^{e}ncias bibliogr\'{a}ficas conforme as regras da ABNT s\~{a}o um
dos principais objetivos do \abnTeX. Consulte os manuais
\citeonline{abntex2cite} e \citeonline{abntex2cite-alf} para obter informa\c{c}\~{o}es
sobre como utilizar as refer\^{e}ncias bibliogr\'{a}ficas.

%-
\subsection{Acentua\c{c}\~{a}o de refer\^{e}ncias bibliogr\'{a}ficas}
%-

Normalmente n\~{a}o h\'{a} problemas em usar caracteres acentuados em arquivos
bibliogr\'{a}ficos (\texttt{*.bib}). Por\'{e}m, como as regras da ABNT fazem uso quase
abusivo da convers\~{a}o para letras mai\'{u}sculas, \'{e} preciso observar o modo como se
escreve os nomes dos autores. Na ~\autoref{tabela-acentos} voc\^{e} encontra alguns
exemplos das convers\~{o}es mais importantes. Preste aten\c{c}\~{a}o especial para `\c{c}' e `\'{\i}'
que devem estar envoltos em chaves. A regra geral \'{e} sempre usar a acentua\c{c}\~{a}o
neste modo quando houver convers\~{a}o para letras mai\'{u}sculas.

\begin{table}[htbp]
\caption{Tabela de convers\~{a}o de acentua\c{c}\~{a}o.}
\label{tabela-acentos}

\begin{center}
\begin{tabular}{ll}\hline\hline
acento & \textsf{bibtex}\\
\`{a} \'{a} \~{a} & \verb+\`a+ \verb+\'a+ \verb+\~a+\\
\'{\i} & \verb+\'{\i}+\\
\c{c} & \verb+\c{c}+\\
\hline\hline
\end{tabular}
\end{center}
\end{table}


% ---
\section{Precisa de ajuda?}
% ---

Consulte a FAQ com perguntas frequentes e comuns no portal do \abnTeX:
\url{https://github.com/abntex/abntex2/wiki/FAQ}.

Inscreva-se no grupo de usu\'{a}rios \LaTeX:
\url{http://groups.google.com/group/latex-br}, tire suas d\'{u}vidas e ajude
outros usu\'{a}rios.

Participe tamb\'{e}m do grupo de desenvolvedores do \abnTeX:
\url{http://groups.google.com/group/abntex2} e fa\c{c}a sua contribui\c{c}\~{a}o \`{a}
ferramenta.

% ---
\section{Voc\^{e} pode ajudar?}
% ---

Sua contribui\c{c}\~{a}o \'{e} muito importante! Voc\^{e} pode ajudar na divulga\c{c}\~{a}o, no
desenvolvimento e de v\'{a}rias outras formas. Veja como contribuir com o \abnTeX\
em \url{https://github.com/abntex/abntex2/wiki/Como-Contribuir}.

% ---
\section{Quer customizar os modelos do \abnTeX\ para sua institui\c{c}\~{a}o ou
universidade?}
% ---

Veja como customizar o \abnTeX\ em:
\url{https://github.com/abntex/abntex2/wiki/ComoCustomizar}.

