%% abtex2-modelo-include-comandos.tex, v-1.9.6 laurocesar
%% Copyright 2012-2016 by abnTeX2 group at http://www.abntex.net.br/
%%
%% This work may be distributed and/or modified under the
%% conditions of the LaTeX Project Public License, either version 1.3
%% of this license or (at your option) any later version.
%% The latest version of this license is in
%%   http://www.latex-project.org/lppl.txt
%% and version 1.3 or later is part of all distributions of LaTeX
%% version 2005/12/01 or later.
%%
%% This work has the LPPL maintenance status `maintained'.
%%
%% The Current Maintainer of this work is the abnTeX2 team, led
%% by Lauro C\'{e}sar Araujo. Further information are available on
%% http://www.abntex.net.br/
%%
%% This work consists of the files abntex2-modelo-include-comandos.tex
%% and abntex2-modelo-img-marca.pdf
%%

% ---
% Este cap\'{\i}tulo, utilizado por diferentes exemplos do abnTeX2, ilustra o uso de
% comandos do abnTeX2 e de LaTeX.
% ---

\chapter{Desenvolvimento}

% ---
\section{Tarefa I}
% ---

A codifica\c{c}\~{a}o de todos os arquivos do \abnTeX\ \'{e} \texttt{UTF8}. \'{E} necess\'{a}rio que
voc\^{e} utilize a mesma codifica\c{c}\~{a}o nos documentos que escrever, inclusive nos
arquivos de base bibliogr\'{a}ficas |.bib|.

\subsection{Procedimento}
\subsection{Resultado}

% ---
\section{Tarefa II}

\subsection{Procedimento}
\subsection{Resultado}

% ---

\index{cita\c{c}\~{o}es!diretas}Utilize o ambiente \texttt{citacao} para incluir
cita\c{c}\~{o}es diretas com mais de tr\^{e}s linhas:

\begin{citacao}
As cita\c{c}\~{o}es diretas, no texto, com mais de tr\^{e}s linhas, devem ser
destacadas com recuo de 4 cm da margem esquerda, com letra menor que a do texto
utilizado e sem as aspas. No caso de documentos datilografados, deve-se
observar apenas o recuo \cite[5.3]{NBR10520:2002}.
\end{citacao}

Use o ambiente assim:

\begin{verbatim}
\begin{citacao}
As cita\c{c}\~{o}es diretas, no texto, com mais de tr\^{e}s linhas [...] deve-se observar
apenas o recuo \cite[5.3]{NBR10520:2002}.
\end{citacao}
\end{verbatim}

O ambiente \texttt{citacao} pode receber como par\^{a}metro opcional um nome de
idioma previamente carregado nas op\c{c}\~{o}es da classe (\autoref{sec-hifenizacao}). Nesse
caso, o texto da cita\c{c}\~{a}o \'{e} automaticamente escrito em it\'{a}lico e a hifeniza\c{c}\~{a}o \'{e}
ajustada para o idioma selecionado na op\c{c}\~{a}o do ambiente. Por exemplo:

\begin{verbatim}
\begin{citacao}[english]
Text in English language in italic with correct hyphenation.
\end{citacao}
\end{verbatim}

Tem como resultado:

\begin{citacao}[english]
Text in English language in italic with correct hyphenation.
\end{citacao}

\index{cita\c{c}\~{o}es!simples}Cita\c{c}\~{o}es simples, com at\'{e} tr\^{e}s linhas, devem ser
inclu\'{\i}das com aspas. Observe que em \LaTeX as aspas iniciais s\~{a}o diferentes das
finais: ``Amor \'{e} fogo que arde sem se ver''.

% ---
\section{Tarefa III}
% ---

As notas de rodap\'{e} s\~{a}o detalhadas pela NBR 14724:2011 na se\c{c}\~{a}o 5.2.1\footnote{As
notas devem ser digitadas ou datilografadas dentro das margens, ficando
separadas do texto por um espa\c{c}o simples de entre as linhas e por filete de 5
cm, a partir da margem esquerda. Devem ser alinhadas, a partir da segunda linha
da mesma nota, abaixo da primeira letra da primeira palavra, de forma a destacar
o expoente, sem espa\c{c}o entre elas e com fonte menor
\citeonline[5.2.1]{NBR14724:2011}.}\footnote{Caso uma s\'{e}rie de notas sejam
criadas sequencialmente, o \abnTeX\ instrui o \LaTeX\ para que uma v\'{\i}rgula seja
colocada ap\'{o}s cada n\'{u}mero do expoente que indica a nota de rodap\'{e} no corpo do
texto.}\footnote{Verifique se os n\'{u}meros do expoente possuem uma v\'{\i}rgula para
dividi-los no corpo do texto.}.

\subsection{Procedimento}
\subsection{Resultado}


% ---

