\chapter*[Introdu\c{c}\~{a}o]{Introdu\c{c}\~{a}o}
\addcontentsline{toc}{chapter}{Introdu\c{c}\~{a}o}
Neste relatório encontra-se o comparativo entre resultados teóricos e resultados experimentais de circuitos em conexão emissor comum.
Tal conexão se dá quando um transistor bipolar é ligado em série com um elemento de carga, no caso, um resistor. Sua denominação se refere ao terminal do emissor do transistor que é conectado à 0 volts ou à terra, já o terminal do coletor é conectado à carga da saída, e o terminal da base atua como a entrada de sinal. Os circuitos emissor comum são geralmente utilizados para amplificar sinais de baixa tensão.\\
A análise de erro foi realizada analisando medidas características do circuito, as suas tensões e intensidade de correntes e o quanto eram divergentes dos valores teóricos. No mais, constam no relatório definições básicas dos circuitos elétricos.
