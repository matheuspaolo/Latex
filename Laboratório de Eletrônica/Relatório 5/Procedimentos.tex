\setcounter{topnumber}{5}
\setcounter{bottomnumber}{5}
\setcounter{totalnumber}{5}

\chapter{Procedimentos e resultados}

Após a implementação do circuito, obtiveram-se os seguintes valores:

\centerline{\begin{minipage}[c]{\textwidth}
		\centering
		\noindent
		\captionof{table}{Comparação entre os valores teóricos e experimentais}
		\begin{tabular}{cccc}
	\toprule
	Variável & Valor teórico & Valor prático & Erro (\%) \\
	\midrule \midrule
	$V_{CE}$ & 14,01 V & 14,21 V & 1,40 \\
	\midrule
	$V_{BE}$ & 0,7 V & 0,629 V & 10,14 \\
	\midrule
	$V_{BC}$ & -13,31 V & 13,55 V & 1,77 \\
	\midrule
	$I_{C}$ & 9,86 mA & 10,15 mA & 2,96 \\
	\midrule
	$I_{B}$ & 30,42 $\mu A$ & 31,3 $\mu A$ & 2,87 \\
	\midrule
	$I_{E}$ & 9,89 mA & 10,15 mA & 2,85 \\
	\midrule
	$\beta$ & 324 &  324,28 & 0,09 \\
	\midrule
	\bottomrule
\end{tabular}%
		\legend{Fonte: Produzido pelos autores}
		\label{}
\end{minipage}}

Essa margem de erro do valor teórico para o valor prático ocorre, pelas condições de operação do transistor, variando assim os valores e pela questões teóricas, pois usamos formulas aproximadas, onde percebemos que com exceção da $ V_{BE} $, que é um valor teórico determinado sendo assim uma maior percentagem de erro, o restante dos cálculos foram muitos próximos do valor prático, tendo assim ao calcular valores teóricos, estaremos próximos dos valores de operação do transistor, levando em consideração as condições de operação.