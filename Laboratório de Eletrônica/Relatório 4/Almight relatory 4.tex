%% abtex2-modelo-relatorio-tecnico.tex, v-1.9.6 laurocesar
%% Copyright 2012-2016 by abnTeX2 group at http://www.abntex.net.br/
%%
%% This work may be distributed and/or modified under the
%% conditions of the LaTeX Project Public License, either version 1.3
%% of this license
%% The latest version of this license is in
%%   http://www.latex-project.org/lppl.txt
%% and version 1.3 or later is part of all distributions of LaTeX
%% version 2005/12/01 or later.
%%
%% This work has the LPPL maintenance status `maintained'.
%%
%% The Current Maintainer of this work is the abnTeX2 team, led
%% by Lauro C\'{e}sar Araujo. Further information are available on
%% http://www.abntex.net.br/
%%
%% This work consists of the files abntex2-modelo-relatorio-tecnico.tex,
%% abntex2-modelo-include-comandos and abntex2-modelo-references.bib
%%

% ------------------------------------------------------------------------
% ------------------------------------------------------------------------
% abnTeX2: Modelo de Relat\'{o}rio T\'{e}cnico/Acad\^{e}mico em conformidade com
% ABNT NBR 10719:2015 Informa\c{c}\~{a}o e documenta\c{c}\~{a}o - Relat\'{o}rio t\'{e}cnico e/ou
% cient\'{\i}fico - Apresenta\c{c}\~{a}o
% ------------------------------------------------------------------------
% ------------------------------------------------------------------------

\documentclass[
	% -- op\c{c}\~{o}es da classe memoir --
	12pt,				% tamanho da fonte
	openright,			% cap\'{\i}tulos come\c{c}am em p\'{a}g \'{\i}mpar (insere p\'{a}gina vazia caso preciso)
	oneside,			% para impress\~{a}o em recto e verso. Oposto a oneside
	a4paper,			% tamanho do papel.
	% -- op\c{c}\~{o}es da classe abntex2 --
	%chapter=TITLE,		% t\'{\i}tulos de cap\'{\i}tulos convertidos em letras mai\'{u}sculas
	%section=TITLE,		% t\'{\i}tulos de se\c{c}\~{o}es convertidos em letras mai\'{u}sculas
	%subsection=TITLE,	% t\'{\i}tulos de subse\c{c}\~{o}es convertidos em letras mai\'{u}sculas
	%subsubsection=TITLE,% t\'{\i}tulos de subsubse\c{c}\~{o}es convertidos em letras mai\'{u}sculas
	% -- op\c{c}\~{o}es do pacote babel --
	english,			% idioma adicional para hifeniza\c{c}\~{a}o
	french,				% idioma adicional para hifeniza\c{c}\~{a}o
	spanish,			% idioma adicional para hifeniza\c{c}\~{a}o
	brazil,				% o \'{u}ltimo idioma \'{e} o principal do documento
	]{abntex2}


% ---
% PACOTES
% ---

% ---
% Pacotes fundamentais
% ---
\usepackage{lmodern}			% Usa a fonte Latin Modern
\usepackage[T1]{fontenc}		% Selecao de codigos de fonte.
\usepackage[utf8]{inputenc}		% Codificacao do documento (convers\~{a}o autom\'{a}tica dos acentos)
\usepackage{indentfirst}		% Indenta o primeiro par\'{a}grafo de cada se\c{c}\~{a}o.
\usepackage{color}				% Controle das cores
\usepackage{graphicx}			% Inclus\~{a}o de gr\'{a}ficos
\usepackage{microtype} 			% para melhorias de justifica\c{c}\~{a}o
\usepackage{float}
\usepackage{adjustbox}
\usepackage{mwe} % new package from Martin scharrer
\usepackage{caption}
\usepackage{cellspace}
\usepackage{gensymb}

\setlength\cellspacetoplimit{4pt}
\setlength\cellspacebottomlimit{4pt}



\usepackage{adjustbox}
\restylefloat{table}

\setcounter{topnumber}{5}
\setcounter{bottomnumber}{5}
\setcounter{totalnumber}{5}
\usepackage[section]{placeins}
% ---

% ---
% Pacotes adicionais, usados no anexo do modelo de folha de identifica\c{c}\~{a}o
% ---
\usepackage{multicol}
\usepackage{multirow}
% ---
	
% ---
% Pacotes adicionais, usados apenas no \^{a}mbito do Modelo Can\^{o}nico do abnteX2
% ---
\usepackage{lipsum}				% para gera\c{c}\~{a}o de dummy text
% ---

% ---
% Pacotes de cita\c{c}\~{o}es
% ---
\usepackage[brazilian,hyperpageref]{backref}	 % Paginas com as cita\c{c}\~{o}es na bibl
\usepackage[alf]{abntex2cite}	% Cita\c{c}\~{o}es padr\~{a}o ABNT
\usepackage[utf8]{inputenc}
\usepackage{amsmath,amsfonts,amssymb}

% ---
% CONFIGURA\c{C}\~{O}ES DE PACOTES
% ---

% ---
% Configura\c{c}\~{o}es do pacote backref
% Usado sem a op\c{c}\~{a}o hyperpageref de backref
\renewcommand{\backrefpagesname}{Citado na(s) p\'{a}gina(s):~}
% Texto padr\~{a}o antes do n\'{u}mero das p\'{a}ginas
\renewcommand{\backref}{}
% Define os textos da cita\c{c}\~{a}o
\renewcommand*{\backrefalt}[4]{
	\ifcase #1 %
		Nenhuma cita\c{c}\~{a}o no texto.%
	\or
		Citado na p\'{a}gina #2.%
	\else
		Citado #1 vezes nas p\'{a}ginas #2.%
	\fi}%
% ---

% ---
% Informa\c{c}\~{o}es de dados para CAPA e FOLHA DE ROSTO
% ---
\titulo{Relat\'{o}rio IV}
\autor{Francisco Edson Birimba Brito \\ Gisele Ribeiro Gomes \\ Gabriel Marques de Silva Abreu \\ Matheus Paolo dos Anjos Mour\~{a}o \\ Paulo Chaves dos Santos J\'{u}nior}
\local{Rio Branco, Acre}
\data{2017}
\instituicao{%
  Universidade Federal do Acre - UFAC
  \par
  Bacharelado em Engenharia El\'{e}trica
  \par
  Laborat\'{o}rio de Eletr\^{o}nica I}
\tipotrabalho{Relat\'{o}rio t\'{e}cnico}
% O preambulo deve conter o tipo do trabalho, o objetivo,
% o nome da institui\c{c}\~{a}o e a \'{a}rea de concentra\c{c}\~{a}o
\preambulo{Relat\'{o}rio de Laborat\'{o}rio de Eletr\^{o}nica I, entregue para a composi\c{c}\~{a}o parcial da nota da N1.
Orientador : Elmer Osman Hancco}
% ---

% ---
% Configura\c{c}\~{o}es de apar\^{e}ncia do PDF final

% alterando o aspecto da cor azul
\definecolor{blue}{RGB}{41,5,195}

% informa\c{c}\~{o}es do PDF
\makeatletter
\hypersetup{
     	%pagebackref=true,
		pdftitle={\@title},
		pdfauthor={\@author},
    	pdfsubject={\imprimirpreambulo},
	    pdfcreator={LaTeX with abnTeX2},
		pdfkeywords={abnt}{latex}{abntex}{abntex2}{relat\'{o}rio t\'{e}cnico},
		colorlinks=true,       		% false: boxed links; true: colored links
    	linkcolor=black,          	% color of internal links
    	citecolor=blue,        		% color of links to bibliography
    	filecolor=magenta,      		% color of file links
		urlcolor=blue,
		bookmarksdepth=4
}
\makeatother
% ---

% ---
% Espa\c{c}amentos entre linhas e par\'{a}grafos
% ---

% O tamanho do par\'{a}grafo \'{e} dado por:
\setlength{\parindent}{1.3cm}

% Controle do espa\c{c}amento entre um par\'{a}grafo e outro:
\setlength{\parskip}{0.2cm}  % tente tamb\'{e}m \onelineskip

% ---
% compila o indice
% ---
\makeindex
% ---

% ----
% In\'{\i}cio do documento
% ----
\begin{document}

% Seleciona o idioma do documento (conforme pacotes do babel)
%\selectlanguage{english}
\selectlanguage{brazil}

% Retira espa\c{c}o extra obsoleto entre as frases.
\frenchspacing

% ----------------------------------------------------------
% ELEMENTOS PR\'{E}-TEXTUAIS
% ----------------------------------------------------------
% \pretextual

% ---
% Capa
% ---
\imprimircapa
% ---

% ---
% Folha de rosto
% (o * indica que haver\'{a} a ficha bibliogr\'{a}fica)
% ---
\imprimirfolhaderosto*
% ---

% ---
% Anverso da folha de rosto:
% ---

% ---
% Agradecimentos

% ---

% ---
% RESUMO
% ---

% resumo na l\'{\i}ngua vern\'{a}cula (obrigat\'{o}rio)
\setlength{\absparsep}{18pt} % ajusta o espa\c{c}amento dos par\'{a}grafos do resumo
 \begin{resumo}
 Segundo a \citeonline[3.1-3.2]{NBR6028:2003}, o resumo deve ressaltar o
 objetivo, o m\'{e}todo, os resultados e as conclus\~{o}es do documento. A ordem e a extens\~{a}o
 destes itens dependem do tipo de resumo (informativo ou indicativo) e do
 tratamento que cada item recebe no documento original. O resumo deve ser
 precedido da refer\^{e}ncia do documento, com exce\c{c}\~{a}o do resumo inserido no
 pr\'{o}prio documento. (\ldots) As palavras-chave devem figurar logo abaixo do
 resumo, antecedidas da express\~{a}o Palavras-chave:, separadas entre si por
 ponto e finalizadas tamb\'{e}m por ponto.

 \noindent
 \textbf{Palavras-chaves}: latex. abntex. editora\c{c}\~{a}o de texto.
\end{resumo}
% ---

 \setlength{\absparsep}{18pt} % ajusta o espa\c{c}amento dos par\'{a}grafos do resumo
\begin{resumo}[Abstract]
This report studied and implemented the application of diodes and transformers for grinding circuits, as well as the collection and analysis of different forms of waves generated by these circuits. Simulations were also performed through the Multisim software for comparison between the experimental values and the values obtained in the simulation.
 \noindent
 \textbf{Keyword}:diode, rectificier circuit, multisim
 \end{resumo} 

% ---
% inserir lista de ilustra\c{c}\~{o}es
% ---
\pdfbookmark[0]{\listfigurename}{lof}
\listoffigures*
\cleardoublepage
% ---

% ---
% inserir lista de tabelas
% ---
% ---
% inserir o sumario
% ---
\pdfbookmark[0]{\contentsname}{toc}
\tableofcontents*
\cleardoublepage
% ---

\textual
\chapter*[Modelagem matemática: o que é?]{Modelagem matemática: o que é?}
\addcontentsline{toc}{chapter}{Modelagem matemática: o que é?}
A modelagem matemática é uma área de conhecimento que estuda a simulação de sistemas e situações reais, com o objetivo de prever como deve será o comportamento e o resultado dos mesmos. Abrange várias áreas de estudo, como física, biologia, engenharia, química, entre outros. Umas das formas que continuam sendo muito utilizadas para a modelagem desses problemas, é a partir das equações diferenciais.
\setcounter{topnumber}{5}
\setcounter{bottomnumber}{5}
\setcounter{totalnumber}{5}

\chapter{Procedimentos e resultados}

Após a implementação do circuito, obtiveram-se os seguintes valores:

\centerline{\begin{minipage}[c]{\textwidth}
		\centering
		\noindent
		\captionof{table}{Comparação entre os valores teóricos e experimentais}
		\begin{tabular}{cccc}
	\toprule
	Variável & Valor teórico & Valor prático & Erro (\%) \\
	\midrule \midrule
	$V_{CE}$ & 14,01 V & 14,21 V & 1,40 \\
	\midrule
	$V_{BE}$ & 0,7 V & 0,629 V & 10,14 \\
	\midrule
	$V_{BC}$ & -13,31 V & 13,55 V & 1,77 \\
	\midrule
	$I_{C}$ & 9,86 mA & 10,15 mA & 2,96 \\
	\midrule
	$I_{B}$ & 30,42 $\mu A$ & 31,3 $\mu A$ & 2,87 \\
	\midrule
	$I_{E}$ & 9,89 mA & 10,15 mA & 2,85 \\
	\midrule
	$\beta$ & 324 &  324,28 & 0,09 \\
	\midrule
	\bottomrule
\end{tabular}%
		\legend{Fonte: Produzido pelos autores}
		\label{}
\end{minipage}}

Essa margem de erro do valor teórico para o valor prático ocorre, pelas condições de operação do transistor, variando assim os valores e pela questões teóricas, pois usamos formulas aproximadas, onde percebemos que com exceção da $ V_{BE} $, que é um valor teórico determinado sendo assim uma maior percentagem de erro, o restante dos cálculos foram muitos próximos do valor prático, tendo assim ao calcular valores teóricos, estaremos próximos dos valores de operação do transistor, levando em consideração as condições de operação.
\chapter{Conclus\~{a}o}

\postextual
\bibliography{Referencias}

\end{document}
